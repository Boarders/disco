Evaluating projects such as this one which develop new educational
tools seems inherently difficult: sample sizes are small, and it is
hard to directly compare learning outcomes across courses taught with
and without the new tool, since the existence of the tool radically
alters the curriculum of the course.  Our best option will be to focus
on qualitative survey data: how do students perceive their learning in
the course?  How do they perceive the helpfulness of the tool?  How do
instructors perceive their students' learning with and without the
tool?

All surveys used in this project will be professionally designed by
the Augusta University survey team, who have experience designing
survey instruments specifically for educational research.

\textbf{Phase One.} The PI Harley Eades has taught discrete
mathematics to computer science students twice a year for the past two
years and has identified many of the topics his students struggle with
each semester.  In this phase he will pilot the curriculum created by
this project.  Following its adoption, data will be collected on
student performance on quizzes and exams, and at some points during
and/or after the semester, students will be surveyed on how they
thought the \thelang{} environment helped them.

The Hendrix College Mathematics and Computer Science department
teaches discrete mathematics to a mixture of mathematics and computer
science students.  In addition, the course is generally taught by a
mathematician rather than a computer scientist.  In Phase One their
course will adopt the curriculum created by this project as well,
co-taught in this first instance by a mathematician and PI Brent
Yorgey.  Their students will be similarly evaluated.  Furthermore, we
will survey the other instructor to obtain data on the effectiveness
of this project's adoption by faculty with little computer science
background.

\textbf{Phase two.} As part of this proposal we are asking for funds
to be used for stipends to help recruit discrete mathematics
instructors external to our universities to pilot the curriculum
developed by this project. As a condition of accepting the stipend,
the external instructors will agree to administer similar surveys to
their students, and also complete surveys themselves.

In a sense, we would consider this project at least somewhat
successful if surveyed students have a positive attitude towards the
programming environment.  Sometimes, student attitude and buy-in is
half the battle.  However, it would not be of much use if students
feel positively about the system but do not actually learn much from
it.  We would also hope to see data from instructor surveys indicating
that they believe student learning outcomes are improved with the use
of the system.

%% In addition to the recruitment of external faculty we will be conduct
%% controlled studies at our institutions.  The study will consist of two
%% groups: one that will use \thelang{}, and one that does not.  Then
%% each group will attend a lecture that teaches the students topics
%% related to functions and the \thelang{} environment.  Following the
%% lecture be asked to solve several exercises on what they learned
%% during the lecture.  We will then compare the results of these two
%% groups.

%% BAY: I don't like this controlled experiment thing.  At my
%% institution we don't have enough sections of the course to do this
%% at all (only one section of the course is taught once a year).  And
%% even if there were, I don't like the idea of having to teach two
%% versions of the course which are similar to each other but one
%% using DISCO and one not.  And in any case the sample size is likely
%% to be so small that any sort of comparison would be pointless.
