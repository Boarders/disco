The impact of this project can be split into two important areas:
undergraduate education and undergraduate research.

\textbf{Undergraduate Education.}  \thelang{} will allow the typical
discrete mathematics curriculum to be reshaped to better accommodate
the mixture of mathematics and computer science students in such
courses.  As we explained in
Section~\ref{subsec:reshaping_the_discrete_mathematics_curriculum},
the curriculum can become more active with the students doing in class
programming exercises to reinforce the concepts being introduced.
Furthermore, course can be more well-balanced with respect to the
strengths of the students, e.g. the computer science students will
feel comfortable programming, but will be challenged mathematically,
and the mathematics students will feel comfortable mathematically, but
will be challenged in the programming component.

\begin{itemize}
\item undergraduate research---language development
\item collaboration between our universities
\item use as an example in PL classes
\item curriculum will make it possible for others to adopt this model
  as well
\item encourage playing with math and engagement with CS.
\end{itemize}

\subsubsection{Broader Impacts at Hendrix College}
\label{subsec:broader_impacts_at_hendrix_college}
XXX

\subsection{Broader impacts at Augusta University}
\label{subsec:broader_impacts_at_augusta_university}

\subsubsection{Teaching and Mentoring}
\label{subsec:teaching_and_mentoring}

\textbf{Teaching.}  The Discrete Mathematics course at Augusta
University is a computer science course that is taught by a computer
science faculty member.  Furthermore, the students are predominately
computer science majors in their sophomore year.  This implies that
most of the students already have some basic programming experience,
but they often have only a basic understanding of mathematics.  The
language being developed by this project will be used to add a
programming component to this course.  This will both contribute to
their understanding of the mathematics, but also link the theory to
applications in computer science more easily.  Students will then come
away from the course with a better understanding of discrete
mathematics and how it is applied to solve problems in computer
science.

One of the primary aims of this project is to keep the design of the
language as simple as possible, and so the design and implementation
will be used in the curriculum of the junior/senior Programming
Language Concepts course at Augusta University.  Students will get a
chance to see what the complete design of a non-trivial programming
language looks like, but they will also get a chance to extend it with
new features as course projects.

\ \\
\noindent
\textbf{Undergraduate Research.}  The typical Augusta University
student is a U.S. native non-traditional student -- 30\% of
undergraduate students enrolled at Augusta University are
non-traditional students (25 years of age or older) -- military
veterans -- 9\% of undergraduate students at Augusta University report
that they are either military veterans, active military, or are
members of a military family -- and people of minority groups -- 32\%
of undergraduate students at Augusta University are people of minority
groups -- by providing them with rich research
experiences. Undergraduate research opportunities at Augusta
University have been sparse, and so this proposal will contribute to
creating more research possibilities for these types of students.

%% 61\% of all students at AU are female
%% 63\% of undergraduate students at AU are female.

Increasing the number of females entering computer science and
mathematics is an important endeavor that this project has the
potential to contribute to.  In fact, 61\% of all students attending
Augusta University are female, and 63\% of undergraduate students at
Augusta University are female.  This project as a breadth unlike most
consisting of both a core computer science research aspect and a
teaching research aspect.  This combination provides a number of
opportunities that could be used to recruit and engage female computer
science and mathematics students.

Undergraduate students working on a research project like the one
being proposed here should get a chance to do more than the expected
tasks, for example, just writing programs.  We want to include the
students in all aspects of the project including the design and
theoretical aspects.  This will give them a chance to really
experience research.  Furthermore, many of the students that will work
on this project have never collaborated across universities, hence,
including students in project meetings and allowing the students at
Augusta University and Williams College to contribute to the same code
base will give the students at both universities an authentic research
experience; see
Section~\ref{subsec:broader_impacts_of_cross_university_collaboration}. This
project has the potential convince the types of students listed above
to pursue a career in research and go to graduate school when they
would otherwise never think that such a career path was possible for
them.
% subsection broader_impacts_at_augusta_university (end)

\subsection{Broader Impacts of Cross University Collaboration}
\label{subsec:broader_impacts_of_cross_university_collaboration}
Talk about how our students will be impacted by working with other
students across universities.  How will this be setup to encourage
each group to work together?
% subsection broader_impacts_of_cross_university_collaboration (end)
