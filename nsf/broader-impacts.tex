The impact of this project can be split into two important areas:
undergraduate education and undergraduate research.

\textbf{Undergraduate Education.}  \thelang{} will allow the typical
discrete mathematics curriculum to be reshaped to better accommodate
the mixture of mathematics and computer science students in such
courses.  This project will design new curriculum for discrete
mathematics courses that results in the course being more active by
giving the students more opportunities to do in class programming
exercises that reinforce the concepts being introduced.  The
introduction of concepts can be designed with \thelang{} environment
in mind capitalizing on visualizations and interactivity making it
easier to introduce difficult concepts like recursion.  Furthermore,
the course can be more well-balanced with respect to the strengths of
the students, e.g. the computer science students will feel comfortable
programming, but will be challenged mathematically, and the
mathematics students will feel comfortable mathematically, but will be
challenged in the programming component.

This new curriculum also provides an elegant means of including an
additional programming paradigm in computer science programs.  The
functional paradigm is one that forces students to think differently
and provides them a chance to learn a paradigm that is becoming more
important both practically and academically.  Thus, better preparing
students for graduate school and a successful career in the industry.

\thelang{} environment is a playground of sorts.  It can be viewed as
a way for both computer science and mathematics students to play with
mathematical and computational concepts in a way unlike most other
languages.  \thelang{} will be designed from the ground up with
education in mind giving ample opportunity for it to foster
creativity.  \thelang{} will make clever use visualizations and highly
advanced error messages.  This will make it possible for students to
learn quickly from the system itself.  In addition, lots of different
libraries could be developed in \thelang{} in a number of different
subjects, e.g. physics, engineering, machine learning, chemistry, and
many more.  Thus, \thelang{} can be seen as a means of engaging a
large number of different types of students in computer science
topics.

We will make this newly designed curriculum available to the public
making it easier to adopt in other computer science and mathematics
programs.  By hosting workshops and tutorials at well known education
conferences like for example at the annual conference for the ACM
Special Interest Group on Computer Science Education (SIGCSE) we will
provide a means for faculty to become familiar with the \thelang{}
environment making its adoption easier.  Therefore, this project has
the potential to impact many students at multiple institutions.

The design and implementation of \thelang{} will be used in the
curriculum of an undergraduate programming language concepts courses
at both universities.  Homework exercises and projects will be
designed to extend both the design of the language and its
implementation with new features to give students the opportunity to
experience modifying and improving a realistic programming language.

\textbf{Undergraduate Research.} One important aspect of this project
is that it will be largely created by undergraduate research
assistants.  These assistants will have a chance to conduct research
in programming language design and development, the design and
development of web-based integrated development environments, or
education research on the design of discrete mathematics courses using
the \thelang{} environment.  These broad categories open lots of
opportunities for undergraduate research projects.

There will be two main research groups working on this project one at
Augusta University and one at Hendrix College.  The undergraduate
research assistants will have the opportunity to collaborate on
project across both universities.  In addition, both groups will come
together for a yearly project meeting located at one of the
institutions.  These meetings will include the undergraduate research
assistants.  This project will allow for the undergraduates to
experience research in its entirety which could lead to these students
moving onto graduate school.

The remainder of this section discusses the broader impacts of this
project at Hendrix College and Augusta University.

\subsection{Broader Impacts at Hendrix College}
\label{subsec:broader_impacts_at_hendrix_college}
XXX

\subsection{Broader impacts at Augusta University}
\label{subsec:broader_impacts_at_augusta_university}

\textbf{Teaching.}  The Discrete Mathematics course at Augusta
University is a computer science course that is taught by a computer
science faculty member.  Furthermore, the students are predominately
computer science majors in their sophomore year.  This implies that
most of the students already have some basic programming experience,
but they often have only a basic understanding of mathematics.  The
language being developed by this project will be used to add a
programming component to this course.  This will both contribute to
their understanding of the mathematics, but also link the theory to
applications in computer science more easily.  Students will then come
away from the course with a better understanding of discrete
mathematics and how it is applied to solve problems in computer
science.

One of the primary aims of this project is to keep the design of the
language as simple as possible, and so the design and implementation
will be used in the curriculum of the junior/senior Programming
Language Concepts course at Augusta University.  Students will get a
chance to see what the complete design of a non-trivial programming
language looks like, but they will also get a chance to extend it with
new features as course projects.

\ \\
\noindent
\textbf{Undergraduate Research.}  The typical Augusta University
student is a U.S. native non-traditional student -- 30\% of
undergraduate students enrolled at Augusta University are
non-traditional students (25 years of age or older) -- military
veterans -- 9\% of undergraduate students at Augusta University report
that they are either military veterans, active military, or are
members of a military family -- and people of minority groups -- 32\%
of undergraduate students at Augusta University are people of minority
groups -- by providing them with rich research
experiences. Undergraduate research opportunities at Augusta
University have been sparse, and so this proposal will contribute to
creating more research possibilities for these types of students.

%% 61\% of all students at AU are female
%% 63\% of undergraduate students at AU are female.

Increasing the number of females entering computer science and
mathematics is an important endeavor that this project has the
potential to contribute to.  In fact, 61\% of all students attending
Augusta University are female, and 63\% of undergraduate students at
Augusta University are female.  This project as a breadth unlike most
consisting of both a core computer science research aspect and a
teaching research aspect.  This combination provides a number of
opportunities that could be used to recruit and engage female computer
science and mathematics students.

Undergraduate students working on a research project like the one
being proposed here should get a chance to do more than the expected
tasks, for example, just writing programs.  We want to include the
students in all aspects of the project including the design and
theoretical aspects.  This will give them a chance to really
experience research.  Furthermore, many of the students that will work
on this project have never collaborated across universities, hence,
including students in project meetings and allowing the students at
Augusta University and Hendrix College to contribute to the same code
base will give the students at both universities an authentic research
experience.  This project has the potential convince the types of
students listed above to pursue a career in research and go to
graduate school when they would otherwise never think that such a
career path was possible for them.
% subsection broader_impacts_at_augusta_university (end)
