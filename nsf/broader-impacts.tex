The broader impacts of this project can be categorized into two main
areas: undergraduate education and undergraduate research.

\paragraph{Undergraduate Education}  \thelang will allow the typical
discrete mathematics curriculum to be reshaped to better accommodate
the mixture of mathematics and computer science students in such
courses.  This project will design new curriculum for discrete
mathematics courses that results in the course being more active by
giving the students more opportunities to do in class programming
exercises that reinforce the concepts being introduced.  The
introduction of concepts can be designed with \thelang environment
in mind capitalizing on visualizations and interactivity making it
easier to introduce difficult concepts like recursion.  Furthermore,
the course can be more well-balanced with respect to the strengths of
the students, e.g. the computer science students will feel comfortable
programming, but will be challenged mathematically, and the
mathematics students will feel comfortable mathematically, but will be
challenged in the programming component.

This new curriculum also provides an elegant means of including an
additional programming paradigm in computer science programs.  The
functional paradigm is one that forces students to think differently
and provides them a chance to learn a paradigm that is becoming more
important both practically and academically.  Thus, better preparing
students for graduate school and a successful career in the industry.

%% \thelang environment is a playground of sorts.  It can be viewed as
%% a way for both computer science and mathematics students to play with
%% mathematical and computational concepts in a way unlike most other
%% languages.  \thelang will be designed from the ground up with
%% education in mind giving ample opportunity for it to foster
%% creativity.  \thelang will make clever use visualizations and highly
%% advanced error messages.  This will make it possible for students to
%% learn quickly from the system itself.  In addition, lots of different
%% libraries could be developed in \thelang in a number of different
%% subjects, e.g. physics, engineering, machine learning, chemistry, and
%% many more.  Thus, \thelang can be seen as a means of engaging a
%% large number of different types of students in computer science
%% topics.

\thelang and its accompanying curricular modules will be made freely
available, and getting it adopted at other institutions beyond the PI
institutions will be a major focus.  Through online training materials
and workshops at appropriate conferences (for example, the annual
SIGCSE computer science education conference) our aim will be to make
adoption by other institutions as easy as possible.  Therefore, this
project has the potential to impact many students at multiple
institutions across the country.

The design and implementation of \thelang will be used as a major case
study in the curriculum of programming language concepts courses at
both universities.  Homework exercises and projects will be designed
to extend both the design of the language and its implementation with
new features to give students the opportunity to experience modifying
and improving a realistic programming language.  Although getting
other institutions to adopt \thelang as a case study in their own
programming languages courses will not be an explicit goal of this
project, the course materials will be made freely available online as
well.

\paragraph{Undergraduate Research} One important aspect of this project
is that it will be largely created by undergraduate research
assistants.  These assistants will have a chance to conduct research
in programming language design and development, the design and
development of web-based integrated development environments, or
education research on the design of discrete mathematics courses using
the \thelang environment.  These broad categories open lots of
opportunities for undergraduate research projects.

There will be two main research groups working on this project one at
Augusta University and one at Hendrix College.  The undergraduate
research assistants will have the opportunity to collaborate on
project across both universities.  In addition, both groups will come
together for a yearly project meeting located at one of the
institutions.  These meetings will include the undergraduate research
assistants.  This project will allow for the undergraduates to
experience research in its entirety which could lead to these students
moving onto graduate school.

The remainder of this section discusses the broader impacts of this
project at Augusta University and Hendrix College.

\subsection{Broader impacts at Augusta University}
\label{subsec:broader_impacts_at_augusta_university}

\paragraph{Undergraduate Education}  The discrete mathematics course at Augusta
University is a computer science course that is taught by a computer
science faculty member.  Furthermore, the students are predominately
computer science majors in their sophomore year.  This implies that
most of the students already have some basic programming experience,
but they often have only a basic understanding of mathematics.  The
language being developed by this project will be used to add a
programming component to this course.  This will both contribute to
their understanding of the mathematics, but also link the theory to
applications in computer science more easily.  Students will then come
away from the course with a better understanding of discrete
mathematics and how it is applied to solve problems in computer
science.

One of the primary aims of this project is to keep the design of the
language as simple as possible, and so the design and implementation
will be used in the curriculum of the junior/senior Programming
Language Concepts course at Augusta University.  Students will get a
chance to see what the complete design of a non-trivial programming
language looks like, but they will also get a chance to extend it with
new features as course projects.

\paragraph{Undergraduate Research}  The typical Augusta University
student is a U.S. native non-traditional student: 30\% of
undergraduate students enrolled at Augusta University are
non-traditional students (25 years of age or older); 9\% report that
they are either military veterans, active military, or are members of
a military family; and 32\% are members of minority groups.
Undergraduate research opportunities at Augusta University have been
sparse, and this proposal will contribute towards creating more
research possibilities for these types of students.

%% 61\% of all students at AU are female
%% 63\% of undergraduate students at AU are female.

Increasing the number of females entering computer science and
mathematics is an important endeavor that this project has the
potential to contribute to.  In fact, 61\% of all students attending
Augusta University are female, and 63\% of undergraduate students at
Augusta University are female.  This project as a breadth unlike most
consisting of both a core computer science research aspect and a
teaching research aspect.  This combination provides a number of
opportunities that could be used to recruit and engage female computer
science and mathematics students.

Undergraduate students working on a research project like the one
being proposed here should get a chance to do more than the expected
tasks, for example, just writing programs.  We want to include the
students in all aspects of the project including the design and
theoretical aspects.  This will give them a chance to really
experience research.  Furthermore, many of the students that will work
on this project have never collaborated across universities, hence,
including students in project meetings and allowing the students at
Augusta University and Hendrix College to contribute to the same code
base will give the students at both universities an authentic research
experience.  This project has the potential convince the types of
students listed above to pursue a career in research and go to
graduate school when they would otherwise never think that such a
career path was possible for them.
% subsection broader_impacts_at_augusta_university (end)

\subsection{Broader Impacts at Hendrix College}
\label{subsec:broader_impacts_at_hendrix_college}

\paragraph{Undergraduate Education}  The discrete mathematics course at Hendrix
College is taught by math faculty, and typically populated by a mix of
math and computer science majors.  Although math and computer science
make up a single joint department, there is not as much
cross-fertilization of ideas and interaction between math and computer
science major students as one might hope.  The hope is that
using \thelang in the discrete math course---which is required for all
computer science majors and also taken by many math majors---can form
the basis for richer interactions between the two domains.

\paragraph{Undergraduate Research}  For a small (around 1400
undergraduates) liberal arts college, Hendrix has an impressive rate
of students involved in research projects in the sciences, many of
whom go on to pursue graduate study.  Historically, however, this has
been confined to physical sciences like chemistry and physics; only
recently has there been more of a push to get computer science
students involved in research and potentially interested in graduate
study.  This project would provide many additional opportunities for
undergraduates to get involved in all aspects of programming language
design and computer science education research.
