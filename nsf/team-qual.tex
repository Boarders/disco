Harley Eades has extensive experience designing and analyzing both
simply typed and dependently typed programming languages and
logics. As part of his dissertation Harley Eades was a member of the
Trellys project \citep{Kimmel:2013,Sjoberg:2012} in which he
contributed to the design of a dependently typed functional
programming language called Separation of Proof from Program (Sep3)
whose design is now detailed in his dissertation
\citep{Eades:2014b,Kimmel:2013}. The second major part of his
dissertation was dedicated to the meta-theoretic analysis of various
type theories. First, he studied a new proof technique for proving
normalization of simply typed and predicative polymorphic type
theories called hereditary substitution, and then proved completeness
of a new logic and simple type theory called dualized simple type
theory \citep{Eades:2014b}.

Since graduate school Harley Eades transitioned into an assistant
professor position at Augusta University where he teaches discrete
mathematics, theory of computation, and programming language
concepts. His research is now focused on the design and meta-theoretic
analysis of programming languages based on linear logic for verifying
security critical systems.  He was awarded a NSF Computer and
Information Science and Engineering (CISE) Research Initiation
Initiative (CRII) grant to develop a new domain-specific functional
programming language for conducting threat analysis (Grant
No. 1565557).

Brent Yorgey has experience in both functional programming and
mathematics education.  He is well-known in the Haskell functional
programming language community for his educational materials, such as
the Typeclassopedia \citep{Typeclassopedia}, and his course
materials for an undergraduate introductory Haskell course developed
at the University of Pennsylvania, both of which are routinely
recommended by the community as some of the best materials for
learning functional programming in Haskell \citep{bitemyapp-learn}.
Yorgey's blog, \emph{The Math Less Traveled} \citep{yorgey-tmlt},
explains discrete mathematics topics to a broad audience and attracts
around 10,000 unique visitors each month.

Yorgey is also the lead developer
of \emph{diagrams} \citep{diagrams-web, Yates:2015}, a
high-level framework for creating vector graphics, which will be used
as the basis for visualizations in the \thelang environment.

Since completing his Ph.D. at the University of Pennsylvania as part
of the Programming Languages research group, he has taken up an
assistant professor position at Hendrix College, where he teaches
courses across a broad range of the computer science curriculum,
including the introductory sequence, algorithms, programming
languages, and functional programming.  His research focuses on
embedded domain-specific languages in Haskell, as well as on the
intersection of functional programming and combinatorics.
