\centerline{\Large NAME OF THE PROJECT GOES HERE}
\vspace{1ex}
\centerline{\Large CONTINUED NAME OF THE PROJECT}
\vspace{3ex}
\centerline{\large PIs: Harley Eades and Brent Yorgey}
\vspace{3ex}

\begin{comment}
  The proposal must contain a summary of the proposed activity suitable
  for publication, not more than one page in length. It should not be an
  abstract of the proposal, but rather a self-contained description of the
  activity that would result if the proposal were funded. The summary
  should be written in the third person and include a statement of
  objectives and methods to be employed. It must clearly address in
  separate statements (within the one-page summary): (1) the intellectual
  merit of the proposed activity; and (2) the broader impacts resulting
  from the proposed activity. (See Chapter III for further descriptive
  information on the NSF merit review criteria.) It should be informative
  to other persons working in the same or related fields and, insofar as
  possible, understandable to a scientifically or technically literate lay
  reader. Proposals that do not separately address both merit review
  criteria within the one page Project Summary will be returned without
  review.
\end{comment}

XXX a couple paragraphs describing problem

{\bf Objectives: } The objective of this project is to...

{\bf Methodology: } The project will...

{\bf Intellectual Merit:} The intellectual merit of the proposed work
includes...

{\bf Broader Impact: } The broader impact of the proposed work
includes...

{\bf Key Words:} Discrete mathematics, Functional programming, XXX


%%% Local Variables:
%%% mode: latex
%%% TeX-master: "proposal"
%%% End: