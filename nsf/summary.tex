\centerline{\Large DISCO: A PROGRAMMING ENVIRONMENT}
\vspace{1ex}
\centerline{\Large FOR DISCRETE MATHEMATICS EDUCATION}
\vspace{3ex}
\centerline{\large PIs: Harley Eades and Brent Yorgey}
\vspace{3ex}

\begin{comment}
  The proposal must contain a summary of the proposed activity suitable
  for publication, not more than one page in length. It should not be an
  abstract of the proposal, but rather a self-contained description of the
  activity that would result if the proposal were funded. The summary
  should be written in the third person and include a statement of
  objectives and methods to be employed. It must clearly address in
  separate statements (within the one-page summary): (1) the intellectual
  merit of the proposed activity; and (2) the broader impacts resulting
  from the proposed activity. (See Chapter III for further descriptive
  information on the NSF merit review criteria.) It should be informative
  to other persons working in the same or related fields and, insofar as
  possible, understandable to a scientifically or technically literate lay
  reader. Proposals that do not separately address both merit review
  criteria within the one page Project Summary will be returned without
  review.
\end{comment}

A course in \emph{discrete mathematics} is typically a core
requirement for undergraduates pursuing a degreee in computer
science. The course introduces mathematical structures and techniques
of foundational importance in computer science, such as induction and
recursion, logic, modular arithmetic, functions, relations, and
graphs. Despite its critical importance for continued success in
computer science, many students struggle to see the relevance of the
course content or to connect it to any of their previous knowledge.

\emph{Functional programming} is a programming paradigm organized
around functions rather than sequences of instructions.  XXX why isn't
it prevalent.  Why it meshes well with math.  But it's seeing
increasing adoption in industry, tackling problems like parallelism.
To prepare students to be productive and engaged, it's critical that
they are introduced to functional programming, but many departments
struggle to fit it in the curriculum.

Combining discrete mathematics and the basics of functional
programming into a single course is a promising solution to both
problems.  The topics complement each other well; students are better
able to see the relevance of discrete mathematics concepts when they
can explore them in a computational way; and functional programming
can be introduced early in the curriculum.  However, doing this comes
with problems of its own. Most existing functional languages are not
designed with teaching in mind, and notation for existing languages
does not always match well with standard mathematical practice.
Moreover, requiring instructors to learn a large, complex,
general-purpose language makes it hard to get buy-in from faculty,
especially when the discrete mathematics course is taught by math
faculty, as it is at many smaller institutions. XXX the right way to
solve these problems is to develop a new functional language and
programming environment, designed from the ground up to support
discrete mathematics education.

{\bf Objectives: } The objective of this project is to develop a
programming environment and curriculum specifically designed to
support students in exploring discrete mathematics and basic
functional programming.

{\bf Methodology: } The project will apply proven programming language
technologies and design principles to develop a programming language
and interactive web-based environment tailored for discrete
mathematics education.

The project will also develop curricular materials using XXX

computation.  Pilot the course.  Get other institutions to beta-test
the course.

{\bf Intellectual Merit:} The intellectual merit of the proposed work
includes...

{\bf Broader Impact: } The broader impact of the proposed work
includes...

The programming language and web-based environment will be made freely
available, and all libraries and source code developed by the project
team will be open source and distributed via standard XXX.

{\bf Key Words:} Discrete mathematics, functional programming


%%% Local Variables:
%%% mode: latex
%%% TeX-master: "proposal"
%%% End:
