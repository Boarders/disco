Using functional programming to teach discrete mathematics goes back
at least to the 1990s.  The discrete mathematics course taught by
Wainwright at the University of Tulsa has always had a programming
component
\cite{Wainwright:1992}, but it was not until non-computer science
majors began taking the course that he was forced to decide whether to
keep the programming part or to drop it.  He decided to keep it, but
change the language to one that is easy to introduce and that
complements mathematics.  A natural solution was functional
programming, and he chose the functional programming language Miranda.
Wainwright reported that he was able to get his students fully up to
speed on programming in Miranda in just one week.

Henderson has used the automated theorem prover Prolog and the
functional programming language Standard ML in discrete mathematics
for over fifteen years \cite{Henderson:2002}.  Prolog is used when the
students study logic, but Standard ML is used when they are learning
functions, recursion, induction, sets, lists, and so on.  He has found
that students find more value in the concepts when they can apply them
in a programming environment.  Henderson's work is similar to the work
of Scharff and Wildenberg \cite{Scharff:2002} who also use Standard ML
to teach discrete mathematics. The project being proposed here differs
from their work in that \thelang{} will be designed specifically with
education in mind, and Prolog and Standard ML were not.

Doets and Van Eijck published a book \cite{Doets:2004} on using the
functional programming language Haskell to teach logic, mathematics,
and programming, and has been used by many institutions.  A similar
book by O'Donnell \emph{et al.} also proposes to use Haskell in discrete
mathematics \cite{ODonnell:2006}. The major difference between
their work and the project being proposed here is that Haskell is a
general purpose programming language, and hence, has many features
that are not required or conducive to education.

Felleisen \emph{et al.} propose a first year introductory sequence
which is uses functional programming
\cite{FELLEISEN_FINDLER_FLATT_KRISHNAMURTHI_2004}, but the authors
warn that the curriculum should be based on a functional programming
language that is designed for education, and one should not use
general purpose programming languages like Haskell, ML, etc.  They
argue that functional programming languages provide a natural
environment for organizing and structuring programs which they claim
are the two most important parts of an introductory computer science
sequence.

In 2007 the Liberal Arts Computer Science Consortium released a new
model curriculum for computer science degrees at liberal arts
institutions \cite{LiberalArtsComputerScienceConsortium:2007}.  They
recommend that the best place to introduce functional programming is
in a discrete mathematics course.  They emphasize that introducing
students to multiple programming paradigms better prepares them for
the real world.  It forces them to get used to thinking differently
which is a positive for all STEM students.

The ACM Special Interest Group on Programming Languages hosted a
workshop on undergraduate programming language curriculum
\cite{Allen:2008}.  Their results emphasized functional programming
due to the rising need to develop parallel algorithms, and the fact
that functional programming forces students to think differently.
They propose that functional programming be required for all computer
science students.

Cong-Cong Xing \cite{Xing:2008} realized his students were not
understanding the difference between the name of a function, and the
image of a function given some input.  The PI's of this project have
also seen this type of confusion among their students.  Xing decided
to use functional programming and the $\lambda$-calculus---the formal
theory at the heart of functional programming---to aid his students
in understanding functions better. Unfortunately, similar to other
approaches, Xing used Standard ML.  We believe that students would have
even better experiences with a programming environment that is
designed for education.

In 2011 VanDrunen argued that functional programming should be a core
part of a well-rounded computer science curriculum
\cite{VanDrunen:2011}.  He feared that based on computer science
curriculum design trends the functional programming paradigm would be
pushed out of undergraduate education. He proposes that the ideal
place to introduce functional programming is in a freshman or
sophomore level discrete mathematics course.  Just as we have argued
here VanDrunen makes the following points:
\begin{itemize}
\item ``Functional programming and discrete mathematics are closely
  related.''  For example, recursion and induction are fundamental to
  both,
\item ``Functional programming illuminates discrete mathematics.'' For
  example, a discrete mathematics course often introduces a precise
  definition of the concept of a function, which allows functions to
  be used as mathematical objects in and of themselves---and this can
  be a hard concept for students to grasp.  A functional programming
  language helps them solidify their understanding of functions by
  providing opportunities to play with higher-order functions
  (\emph{i.e.} functions which take other functions as inputs, or
  return functions as outputs).
\item ``Functional programming motivates discrete mathematics for
  computer science students.'' Many computer science students are not
  strong mathematics students despite the fact that computer science
  and mathematics are so closely related.  Functional programming has
  the ability to help these students connect with the mathematics by
  phrasing it in terms of what the most computer science students are
  excited about, namely, programming.
\end{itemize}
